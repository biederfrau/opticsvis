\documentclass[english]{scrartcl}
\usepackage[T1]{fontenc}
\usepackage[utf8]{inputenc}
\usepackage[scaled=.8]{beramono}
\usepackage{geometry}
\geometry{verbose,tmargin=3cm,bmargin=3cm,lmargin=3cm,rmargin=3cm}
\usepackage[english]{babel}
\usepackage{microtype}
\usepackage[parfill]{parskip}
\usepackage{amsmath}
\usepackage{graphicx}
\usepackage{hyperref}
\usepackage{nameref}
\usepackage{float}
\usepackage{listings}
\usepackage{color}
\usepackage{fancyhdr}
\usepackage{blindtext}

\pagestyle{fancy}
\fancyhf{}
\rhead{Group 11---Biedermann \& Permann}
\lhead{README: $\text{OPTICS}_\text{vis}$}
\cfoot{\thepage}

\newcommand*{\fullref}[1]{\hyperref[{#1}]{\autoref*{#1}~\nameref*{#1}}}

\definecolor{darkgray}{rgb}{0.66, 0.66, 0.66}
\definecolor{asparagus}{rgb}{0.53, 0.66, 0.42}

\lstdefinestyle{s}{
  commentstyle=\color{darkgray},
  keywordstyle=\bfseries,
  morekeywords={},
  stringstyle=\color{asparagus},
  basicstyle=\ttfamily\footnotesize,
  breakatwhitespace=false,
  keepspaces=true,
  numbersep=5pt,
  showspaces=false,
  showstringspaces=false,
}

\lstset{style=s}

\begin{document}

\title{README: $\text{OPTICS}_\text{vis}$}

\author{Group 11}

\maketitle

\section{How to get it running}

Go into the \texttt{app/} folder and start any simple HTTP server capable of
serving files out of the current directory. We have included \texttt{serve.sh}
which uses PHP to serve at \href{http://localhost:8080}{localhost:8080}.

Nothing else.

\section{References}

Everything that we know, we know from \texttt{bl.ocks.org}. The source code
is composed of mainly \texttt{setup\_*} and \texttt{draw\_*} functions.
When we took something from \texttt{bl.ocks.org}, we cited it above the
relevant \texttt{setup\_*} function.

\end{document}
