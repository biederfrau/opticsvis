\documentclass[naustrian]{beamer}
\usepackage[T1]{fontenc}
\usepackage[utf8]{inputenc}
\usepackage[ngerman]{babel}
\usetheme{metropolis}           % Use metropolis theme
\title{VIS: OPTICS$_\text{vis}$}
\date{\today}
\author{Group 11}
\institute{Fakultät für Informatik}
\titlegraphic{\hfill\includegraphics[height=1cm]{img/uni_logo_farbe.pdf}}

\definecolor{asparagus}{rgb}{0.53, 0.66, 0.42}
\definecolor{alizarin}{rgb}{0.82, 0.1, 0.26}

\setbeamersize{description width=0.57cm}

\begin{document}
\maketitle

\begin{frame}{Agenda}
    \setbeamertemplate{section in toc}[sections numbered]
    \tableofcontents
\end{frame}

\section{Project}

\subsection{Motivation}

\begin{frame}{Project motivation}
    \begin{itemize}
        \item OPTICS: density based clustering
            \begin{itemize}
                \item algorithm jumps between points in some order
                \item records jump distances
            \end{itemize}
        \item output somewhat hard to read
            \begin{itemize}
                \item point order
                \item a list of numbers
            \end{itemize}
        \item staple visualization method: the bar chart
    \end{itemize}
    \begin{figure}[h]
        \centering
        \includegraphics[height=.3\textheight]{img/optics-edited}
        \vspace{1em}
        \includegraphics[width=.3\textwidth]{img/optics-edited-points-black}
    \end{figure}
\end{frame}

\begin{frame}{Project definition}
    \begin{itemize}
        \item colorizing helps a lot
        \item but how does it \emph{work}?
        \item how do these numbers relate to the data?
        \item parameterization?
            \begin{itemize}
                \item min pts
                \item eps
            \end{itemize}
        \item[] $\rightarrow$ OPTICS$_\text{vis}$
    \end{itemize}
    \begin{figure}[h]
        \centering
        \includegraphics[height=.3\textheight]{img/optics-edited-color}
        \vspace{1em}
        \includegraphics[width=.3\textwidth]{img/optics-edited-points}
    \end{figure}
\end{frame}

\section{Users and Tasks}

\begin{frame}{Users}
\begin{itemize}
 \item Teachers
            \begin{itemize}
                \item for educational purposes
            \end{itemize}
  \item Researchers
            \begin{itemize}
                \item exploration
                \item testing before practical usage
            \end{itemize}
\item Anyone
            \begin{itemize}
                \item exploration
            \end{itemize}
  \end{itemize}
\end{frame}

\begin{frame}{Tasks}
\begin{itemize}
 \item Exploration
            \begin{itemize}
                \item get a feeling for the algorithm, get to know it
            \end{itemize}
 \item Education
  	\begin{itemize}
                \item learn about the algorithm and how to interpret the output
            \end{itemize}
  \item Testing
            \begin{itemize}
                \item give an idea if the algorithm fits the users problem
                \item see if result/output is satisfactory and useful
            \end{itemize}
  \end{itemize}
\end{frame}

{
\metroset{sectionpage=none}
\section{Demo}
}

\plain{Demo}

\section{Challenges and Problems}
%edit
\begin{frame}{Challenges and Problems}
    \begin{itemize}
        \item slow implementation
        \item some aspects of the visualization rely on running the algorithm repeatedly, locks up the interface
        \item would benefit from backend
        \item hierarchical clusters are meh
    \end{itemize}
\end{frame}

\section{Future Work}
%edit
\begin{frame}{Future Work}
    \begin{itemize}
        \item different similarity/distance measures?
        \item multiple dimensions (to select from, probably no dimensionality reduction)?
            \begin{itemize}
                \item doesn't really help with understanding OPTICS itself
            \end{itemize}
        \item more/revise interaction
    \end{itemize}
\end{frame}

\plain{Thanks for your attention!\\Questions?}

\end{document}
